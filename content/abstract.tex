% ----------------------------------------------------- Sample ---------------------------------------------------

% \lipsum[6-8]


% ----------------------------------------------------- Real ---------------------------------------------------

A new model, based on matched asymptotic analysis, for the simulation of two-dimensional separated flows around structures, has been developed and tested on a wide range of cases.
The fluid is incompressible and the Reynolds number is high.

In the vast outer region far from the body, the viscosity is negligible and the dynamics are tracked by solving the vorticity transport equation in a Lagrangian reference frame.
The resulting method is grid-free and concentrates its points in the regions of steep gradients; it also allows a simple and exact treatment of the far-field boundary conditions.
It it well adapted to the modeling of transport phenomena.
In order to account for the viscous effect, especially capture the vortex shedding, a more complete description, Prandtl's approximation of Navier-Stokes equations, is used in the thin region close to the body surface, the so-called boundary layer.
The matching conditions at the interface between two regions ensure the smooth advection of vorticity from boundary layer to outer region.

Our model can be considered as a generalization of the Kutta-condition model derived from first principles to treat bodies of arbitrary shape.
It is an attempt to capture the qualitative feature of the solution, especially the vortex shedding, and is guided by quantitative arguments, but not a true limit of the Navier-Stokes equations for large Reynolds number flow.
The method appears to incorporate many of the physical mechanisms of separated flows, and the dependence on Reynolds number has been obtained.
